%%%%%%%%%%%%%%%%%%%%%%%%%%%%%%%%%%%%%%%%%%%%%%%%%%%%%%%%%%%%%%%%%%%%%
%
% CSCI 1430 Writeup Template
%
% This is a LaTeX document. LaTeX is a markup language for producing
% documents. Your task is to fill out this
% document, then to compile this into a PDF document.
%
% TO COMPILE:
% > pdflatex thisfile.tex
%
% For references to appear correctly instead of as '??', you must run
% pdflatex twice.
%
% If you do not have LaTeX and need a LaTeX distribution:
% - Departmental machines have one installed.
% - Personal laptops (all common OS): www.latex-project.org/get/
%
% If you need help with LaTeX, please come to office hours.
% Or, there is plenty of help online:
% https://en.wikibooks.org/wiki/LaTeX
%
% Good luck!
% James and the 1430 staff
%
%%%%%%%%%%%%%%%%%%%%%%%%%%%%%%%%%%%%%%%%%%%%%%%%%%%%%%%%%%%%%%%%%%%%%
%
% How to include two graphics on the same line:
%
% \includegraphics[\width=0.49\linewidth]{yourgraphic1.png}
% \includegraphics[\width=0.49\linewidth]{yourgraphic2.png}
%
% How to include equations:
%
% \begin{equation}
% y = mx+c
% \end{equation}
%
%%%%%%%%%%%%%%%%%%%%%%%%%%%%%%%%%%%%%%%%%%%%%%%%%%%%%%%%%%%%%%%%%%%%%%%%%%%%%%%%%%%%%%%%%%%%%%%%

\documentclass[11pt]{article}

\usepackage[english]{babel}
\usepackage[utf8]{inputenc}
\usepackage[colorlinks = true,
            linkcolor = blue,
            urlcolor  = blue]{hyperref}
\usepackage[a4paper,margin=1.5in]{geometry}
\usepackage{stackengine,graphicx}
\usepackage{fancyhdr}
\setlength{\headheight}{15pt}
\usepackage{microtype}
\usepackage{times}
\usepackage{booktabs}

% python code format: https://github.com/olivierverdier/python-latex-highlighting
\usepackage{pythonhighlight}

\frenchspacing
\setlength{\parindent}{0cm} % Default is 15pt.
\setlength{\parskip}{0.3cm plus1mm minus1mm}

\pagestyle{fancy}
\fancyhf{}
\lhead{Homework 5 Numpy Written Question Writeup}
\rhead{CSCI 1430}
\rfoot{\thepage}

\date{}

\title{\vspace{-1cm}Homework 5 Numpy Written Question Writeup}


\begin{document}
\maketitle
\vspace{-2cm}
\thispagestyle{fancy}

\section*{Instructions}
\begin{itemize}
  \item Refer to the \textsc{Project4\_QuestionsTemplate} for information on Q6 of the Project 4 written questions. Please submit this file to Gradescope under the Project 4 Written Code (Numpy)	assignment along with your completed \texttt{model.py}
  \item \textbf{Please make this document anonymous.}
\end{itemize}

\section*{A6}

\begin{enumerate}
    \item What do these numbers tell us about the capacity of the network, the complexity of the two problems, the value of training, and the value of the two different classification approaches?

    \item How well did each model perform on each dataset? Please use this table to structure your response.

    \begin{itemize}
        \item NN on MNIST: xx\% (highest accuracy)
        	\begin{itemize}
        	\item Epoch 0 loss: xx     Accuracy: xx\%
        	\item Epoch 9 loss: xx     Accuracy: xx\%
        	\end{itemize}
        \item NN+SVM on MNIST: xx\% (highest accuracy)
        	\begin{itemize}
        	\item Epoch 0 loss: xx     Accuracy: xx\%
        	\item Epoch 9 loss: xx     Accuracy: xx\%
        	\end{itemize}
        \item NN on SceneRec: xx\% (highest accuracy)
        	\begin{itemize}
        	\item Epoch 0 loss: xx     Accuracy: xx\%
        	\item Epoch 9 loss: xx     Accuracy: xx\%
        	\end{itemize}
        \item NN+SVM on SceneRec: xx\% (highest accuracy)
        	\begin{itemize}
        	\item Epoch 0 loss: xx     Accuracy: xx\%
        	\item Epoch 9 loss: xx     Accuracy: xx\%
    	    s\end{itemize}
    \end{itemize}

\end{enumerate}

\end{document}
